% Define document class
\documentclass[twocolumn]{aastex631}
\usepackage{showyourwork}
\usepackage{amsmath,esint}
% Begin!
\begin{document}

% Title
\title{Semi-analytic transit models for oblate planets in Jax}

% Author list
\author{Shashank Dholakia} \author{Shishir Dholakia}

% Abstract with filler text
\begin{abstract}

\end{abstract}

% Main body with filler text
\section{Introduction}
\label{sec:intro}


\section{Ellipsoidal Model}
\label{sec:model}
The transit model we use is based on the formalism of the \texttt{starry} framework \citep{starry2019}, which provides closed-form expressions for occultations involving circular bodies in projection with general surface maps expressed in terms of spherical harmonics. \citet{dholakia2022} described an extension of the \texttt{starry} framework for an oblate occulted body, such as a rapidly rotating star. Here we present an extension for the case of any occultor which is an ellipse in projection. \citet{starry2019} showed that if the specific intensity $I(x,y)$ on the projected stellar disk is expanded in terms of the spherical harmonics, the expression for the flux during an occultation can be written in a simpler form that is both computationally and numerically easier to evaulate. We summarize the method here, and refer readers to the full text for the details. 

Our goal is to compute the flux observed during an occultation of a star by a body, which can be written as a surface integral of the intensity over the visible region of the star as:

\begin{equation}
   F = \oiint\limits_{\mathrm{S}(x,y)} I(x,y) \ dS
\end{equation}
where the surface $S$ parametrizes the unobscured portion of the stellar disk and $I$ is the specific intensity at a point $(x,y)$ on the projected surface of the star.

We start with a vector representing the stellar intensity in terms of spherical harmonics $\mathbf{y}$, and transform it into Green's basis as in Eq.~21 of \citet{starry2019}. We then have: 

\begin{equation}
   F = \oiint\limits_{\mathrm{S}(x,y)} \mathbf{\tilde{g}}^\mathsf{T}(x,y) \ \mathbf{A}\ \mathbf{R}\ \mathbf{y}\ dS 
\end{equation}
where the vector $\mathbf{y}$, the rotation matrix $\mathbf{R}$ and the change of basis matrix $\mathbf{A}$ are not dependant on $x$ and $y$ and can consequently be pulled out of the integral, leaving only the Green's basis in the surface integral. Using the vector function $\mathbf{G_n}$, defined as the anti-exterior derivative of the $n$th term in Green's basis, we can write:

\begin{equation} \label{eq:greensintegral}
   \oiint\limits_{\mathrm{S}(x,y)} \mathbf{\tilde{g}}_n(x,y)\ dS \\
   = \oint \mathbf{G}_n(x,y) \cdot d\mathbf{r}
\end{equation}
where $\mathbf{r}$ is a vector function along the closed boundary of the region $S(x,y)$. We can then further decompose the integral in Eq.~\ref{eq:greensintegral} into a section along the stellar projected disk and the occultor's projected disk (see Fig.~\ref{fig:integral_bounds}):

\begin{equation} \label{eq:pandq}
    \oint \mathbf{G}(x,y) \cdot d\mathbf{r} = \mathcal{Q}(\mathbf{G}_n) - \mathcal{P}(\mathbf{G}_n)
\end{equation}

From here, we deviate from the \texttt{starry} framework to solve the line integrals around the star and elliptical occultor. First, we apply a rotation by an angle $\theta$ into a frame where the occultor's major axis is aligned with the x-axis. We then must solve for the points of intersection between the star and the occultor. 

\subsection{Integration bounds}
First, it helps to consider the circular case, where the star is parametrized as the unit circle:
\begin{equation} \label{eq:unitcircle}
x^2 + y^2 = 1
\end{equation}
and the planet as an off-center circle with radius $r_{o}$ as:
\begin{equation} \label{eq:circularplanet}
(x-x_o)^2-(y-y_o)^2 = r_o^2
\end{equation}

We can solve for the intersection points by solving for all $(x,y)$ which satisfy both equations, which yields a quadratic equation with either 0, 1 or 2 real solutions.

In the elliptical case, we can modify Eq.~\ref{eq:circularplanet} by deforming the occultor along the y axis by a value $b$, now taking the radius $r_{eq}$ to represent the projected equatorial radius of the occultor. 
\begin{equation} \label{eq:ellipticalplanet}
(x-x_{o})^2-\frac{(y-y_o)^2}{b^2} = r_{eq}^2
\end{equation}

Solving for y in the above equation and then plugging it into Eq.~\ref{eq:unitcircle} yields a quartic polynomial of the form: 
\begin{equation} \label{eq:quarticform}
Ax^4 + Bx^3 + Cx^2 + Dx + E = 0
\end{equation} 
where
\begin{align*}
A &= \frac{b^4 - 2b^2 + 1}{4y_o^2}\\
B &= \frac{-b^4x_o + b^2x_o}{y_o^2}\\
C &= \frac{-b^4r_{eq}^2 + 3b^4x_o^2 + b^2r_{eq}^2 - b^2x_o^2 + b^2y_o^2 + b^2 + y_o^2 - 1}{2y_o^2} \\
D &= \frac{b^4r_{eq}^2x_o - b^4x_o^3 - b^2x_oy_o^2 - b^2x_o}{y_o^2} \\
E &= \frac{b^4r_{eq}^4 - 2b^4r_{eq}^2x_o^2 + b^4x_o^4 - 2b^2r_{eq}^2y_o^2 - 2b^2r_{eq}^2 + 2b^2x_o^2y_o^2 + 2b^2x_o^2 + y_o^4 - 2y_o^2 + 1}{4y_o^2}\\
\end{align*}
We note that other methods of finding the flux in transit of elliptical occultors also require solving for quartic polynomials \citep{rein2023}. 
\subsection{Planet boundary integral}

\begin{figure}[ht!]
    \script{integral_bounds.py}
    \begin{centering}
        \includegraphics[width=\linewidth]{figures/oblate_planet.pdf}
        \caption{Geometry of the problem of computing the flux due to an oblate occultor as presented in this paper. 
        }
        \label{fig:integral_bounds}
    \end{centering}
\end{figure}

\subsection{Limb darkening}
\subsection{Oblate Case}
\subsection{Prolate Case}
\section{Prospects for Detection with JWST and other Instruments}
\label{sec:jwstdetect}
\subsection{Cramer-Rao Recoverability Estimates}
\label{sec:rec}
\subsection{Fisher Forecasting of Existing Planet Population}
\label{sec:fisherforecasting}
\subsection{Impact of Limb Darkening}
\subsection{Impact on Atmospheric Retreivals}
\section{Implementation in \lowercase{\texttt{jax0planet}}}
\label{sec:jax0planet}
\section{Application to [test system]}
\label{sec:lctest}
\section{Discussion and Future Work}
\label{disc}
\subsection{Hierarchical Inference}
\bibliography{bib}

\end{document}
