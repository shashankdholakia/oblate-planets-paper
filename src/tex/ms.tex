% Define document class
\documentclass[twocolumn]{aastex631}
\usepackage{showyourwork}
\usepackage{amsmath,esint}
% Begin!
\begin{document}

% Title
\title{Differentiable transit models for ellipsoidal planets}

% Author list
\author{Shashank Dholakia} \author{Shishir Dholakia}

% Abstract with filler text
\begin{abstract}

\end{abstract}

% Main body with filler text
\section{Introduction} \label{sec:intro}
The contemporary study of exoplanets frequently assumes spherical bodies. While this is often a valid assumption, the deviations from this case are informative and contain valuable information which may be discarded by assuming sphericity.

In Sec.~\ref{sec:model}, we describe a new forward model for light curves of spherical bodies with ellipsoidal occultors. We then describe the implementation of this model in the Python package \texttt{eclipsoid} in Sec.~\ref{sec:eclipsoid}. The package is built using the \texttt{jax} framework, which allows automatic differentiation of the model with respect to its parameters. We describe a use of this in Sec.~\ref{sec:jwstdetect}, where we utilize Fisher information to quantify the detectability of ellipsoidal exoplanets using JWST and other instruments. Lastly, in Sec.~\ref{sec:lctest}, we apply the new model to the JWST white light curve of WASP107b and demonstrate that we can constrain the oblateness and prolateless of the exoplanet. 

This paper has been written using the open-source package \textcolor{red}{\textit{showyourwork!}} for transparency and ease of reproduction. At the end of every figure caption, we link to a GitHub repository containing a Python script used to generate the figure. All data used in this paper is hosted on Zenodo at the following link.

\section{Ellipsoidal Model}
\label{sec:model}
The transit model we use is based on the formalism of the \texttt{starry} framework \citep{starry2019}, which provides closed-form expressions for occultations involving circular bodies in projection with general surface maps expressed in terms of spherical harmonics. \citet{dholakia2022} described an extension of the \texttt{starry} framework for an oblate occulted body, such as a rapidly rotating star. Here we present an extension for the case of any occultor which is an ellipse in projection. \citet{starry2019} showed that if the specific intensity $I(x,y)$ on the projected stellar disk is expanded in terms of the spherical harmonics, the expression for the flux during an occultation can be written in a simpler form that is both computationally and numerically easier to evaulate. We summarize the method here, and refer readers to the full text for the details. 

Our goal is to compute the flux observed during an occultation of a star by a body, which can be written as a surface integral of the intensity over the visible region of the star as:

\begin{equation}
   F = \oiint\limits_{\mathrm{S}(x,y)} I(x,y) \ dS
\end{equation}
where the surface $S$ parametrizes the unobscured portion of the stellar disk and $I$ is the specific intensity at a point $(x,y)$ on the projected surface of the star.

We start with a vector representing the stellar intensity in terms of spherical harmonics $\mathbf{y}$, and transform it into Green's basis as in Eq.~21 of \citet{starry2019}. We then have: 

\begin{equation}
   F = \oiint\limits_{\mathrm{S}(x,y)} \mathbf{\tilde{g}}^\mathsf{T}(x,y) \ \mathbf{A}\ \mathbf{R}\ \mathbf{y}\ dS 
\end{equation}
where the vector $\mathbf{y}$, the rotation matrix $\mathbf{R}$ and the change of basis matrix $\mathbf{A}$ are not dependant on $x$ and $y$ and can consequently be pulled out of the integral, leaving only the Green's basis in the surface integral. Using the vector function $\mathbf{G_n}$, defined as the anti-exterior derivative of the $n$th term in Green's basis, we can write:

\begin{equation} \label{eq:greensintegral}
   \oiint\limits_{\mathrm{S}(x,y)} \mathbf{\tilde{g}}_n(x,y)\ dS \\
   = \oint \mathbf{G}_n(x,y) \cdot d\mathbf{r}
\end{equation}
where $\mathbf{r}$ is a vector function along the closed boundary of the region $S(x,y)$. We can then further decompose the integral in Eq.~\ref{eq:greensintegral} into a section along the stellar projected disk and the occultor's projected disk (see Fig.~\ref{fig:integral_bounds}):

\begin{equation} \label{eq:pandq}
    \oint \mathbf{G}(x,y) \cdot d\mathbf{r} = \mathcal{Q}(\mathbf{G}_n) - \mathcal{P}(\mathbf{G}_n)
\end{equation}

From here, we deviate from the \texttt{starry} framework to solve the line integrals around the star and elliptical occultor. First, we apply a rotation by an angle $\theta$ into a frame where the occultor's major axis is aligned with the x-axis. We then must solve for the points of intersection between the star and the occultor. 

\subsection{Integration bounds}
First, it helps to consider the circular case, where the star is parametrized as the unit circle:
\begin{equation} \label{eq:unitcircle}
x^2 + y^2 = 1
\end{equation}
and the planet as an off-center circle with radius $r_{o}$ as:
\begin{equation} \label{eq:circularplanet}
(x-x_o)^2-(y-y_o)^2 = r_o^2
\end{equation}

We can solve for the intersection points by solving for all $(x,y)$ which satisfy both equations, which yields a quadratic equation with either 0, 1 or 2 real solutions.

In the elliptical case, we can modify Eq.~\ref{eq:circularplanet} by deforming the occultor along the y axis by a value $b$, now taking the radius $r_o$ to represent the projected equatorial radius of the occultor. 
\begin{equation} \label{eq:ellipticalplanet}
(x-x_{o})^2-\frac{(y-y_o)^2}{b^2} = r_o^2
\end{equation}
where we define $b=1-f$. Here we emphasize that the $r_o$ and $f$ refer to the \textit{projected} equatorial radius and oblateness respectively. 
Solving for y in the above equation and then plugging it into Eq.~\ref{eq:unitcircle} yields a quartic polynomial of the form: 
\begin{equation} \label{eq:quarticform}
Ax^4 + Bx^3 + Cx^2 + Dx + E = 0
\end{equation} 
where
\onecolumngrid
\begin{equation} \label{eq:quarticcoeffs}
\begin{aligned}
A &= \frac{b^4 - 2b^2 + 1}{4y_o^2}\\
B &= \frac{-b^4x_o + b^2x_o}{y_o^2}\\
C &= \frac{-b^4r_o^2 + 3b^4x_o^2 + b^2r_o^2 - b^2x_o^2 + b^2y_o^2 + b^2 + y_o^2 - 1}{2y_o^2} \\
D &= \frac{b^4r_o^2x_o - b^4x_o^3 - b^2x_oy_o^2 - b^2x_o}{y_o^2} \\
E &= \frac{b^4r_o^4 - 2b^4r_o^2x_o^2 + b^4x_o^4 - 2b^2r_o^2y_o^2 - 2b^2r_o^2 + 2b^2x_o^2y_o^2 + 2b^2x_o^2 + y_o^4 - 2y_o^2 + 1}{4y_o^2}\\
\end{aligned}
\end{equation}

\twocolumngrid
We note that other methods of finding the flux in transit of elliptical occultors also require solving for quartic polynomials \citep{rein2023}. We solve the roots of this polynomial using eigendecomposition of the companion matrix (see \ref{sec:eclipsoid} for details of the implementation). \textcolor{red}{we need to update this in eclipsoid to use a more numerically precise method}. The solution gives the x values of the intersection points where the projected disk of the star and the occultor coincide. We can then find the corresponding y values by plugging it back into the formula for either the occultor or occulted body. 

\subsection{Star boundary integral}
The first integral $\mathcal{Q}(\mathbf{G}_n)$ is performed around the boundary of the occulted body's projected disk (bolded black border in Fig.~\ref{fig:integral_bounds}.) While the integrand is the same as in \citet{starry2019}, the bounds of the integral are the roots of the quartic polynomial shown in Eq.~\ref{eq:quarticcoeffs}. We compute the angle $\xi$, defined as the angle between the x-axis and a given intersection point, for all the intersection points. We then sort these angles in clockwise order. 

\subsection{Occultor boundary integral}
The occultor (or planet) boundary integral $\mathcal{P}(\mathbf{G}_n)$ also starts with the intersection points. We define an angle $\phi$ to parametrize the bounds of the line integral. This angle, as noted in \citet{dholakia2022}, is defined similarly to an eccentric anomaly; the angle from the semimajor axis of the planet to the perpendicular projection of an intersection point onto the circle bounding the ellipse. Then, for the integrand, we start with the parametric formula for an ellipse:

\begin{align}
    x &= r_o \cos(\phi) + x_o \\
    y &= r_o b \sin(\phi) + y_o
\end{align}
We then plug this into the integrand for $\mathcal{P}(\mathbf{G}_n)$ in Eq.~\ref{eq:pandq} to obtain:
\begin{align}
\mathcal{P}(\mathbf{G}_n) = \int_{\phi}^{2\pi + \phi}[\ G_{ny}(r_o c_\phi + x_o , r_o b s_\phi + y_o) b c_\phi] \ r_o d\phi \\
- \int_{\phi}^{2\pi + \phi}[G_{nx}(r_o c_\phi + x_o, r_o b s_\phi + y_o)s_\phi]\ r_o d\phi
\end{align}
where we write $\sin{(\phi)}$ as $s_\phi$ and $\cos{(\phi)}$ as $c_\phi$ for brevity.
\begin{figure}[ht!]
    \script{integral_bounds.py}
    \begin{centering}
        \includegraphics[width=\linewidth]{figures/oblate_planet.pdf}
        \caption{Geometry of the problem of computing the flux due to an oblate occultor as presented in this paper. 
        }
        \label{fig:integral_bounds}
    \end{centering}
\end{figure}

\subsection{Limb darkening}
\subsection{Oblate Case}
\subsection{Prolate Case}
\section{Implementation in eclipsoid} \label{sec:eclipsoid}
-implementation details:\\
    -API\\
    -root finding (with custom derivatives)\\
    -gauss quad\\
    -plot showing it matches brute force for each term in Gn   \\
\begin{figure}[ht!]
    \script{bruteforce_comparison.py}
    \begin{centering}
        \includegraphics[width=\linewidth]{figures/bruteforce_comparison.pdf}
        \caption{Comparison of the 1D Green's integral to a brute force 2D integral for each term in Green's basis. Residuals are under the expected numerical error in the brute force solution for all terms.
        }
        \label{fig:bruteforce_comparison}
    \end{centering}
\end{figure}

-speed\\
-precision (do we reach machine precision?)\\
-show derivatives, maybe against numerical diff\\
\section{Prospects for Detection with JWST and other Instruments}
\label{sec:jwstdetect}

\subsection{Cramer-Rao Recoverability Estimates}
\label{sec:rec}

\subsection{Fisher Forecasting of Existing Planet Population}
\label{sec:fisherforecasting}
\subsection{Impact of Limb Darkening}
\section{Application to [test system]}
\label{sec:lctest}
\section{Discussion and Future Work}
\label{disc}
\subsection{Hierarchical Inference}
\bibliography{bib}

\end{document}
