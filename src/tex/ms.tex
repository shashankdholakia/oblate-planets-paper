% Define document class
\documentclass[twocolumn]{aastex631}
\usepackage{showyourwork}
\usepackage{amsmath,esint}
% Begin!
\begin{document}

% Title
\title{Semi-analytic transit models for oblate planets in Jax}

% Author list
\author{Shashank Dholakia} \author{Shishir Dholakia}

% Abstract with filler text
\begin{abstract}

\end{abstract}

% Main body with filler text
\section{Introduction}
\label{sec:intro}


\section{Ellipsoidal Model}
\label{sec:model}
The transit model we use is based on the formalism of the \texttt{starry} framework \citep{starry2019}, which provides closed-form expressions for occultations involving circular bodies in projection with general surface maps expressed in terms of spherical harmonics. \citet{dholakia2022} described an extension of the \texttt{starry} framework for an oblate occulted body, such as a rapidly rotating star. Here we present an extension for the case of any occultor which is an ellipse in projection. 

Our goal is to compute the flux observed during an occultation of a star by a body, which can be written as a surface integral of the intensity over the visible (unobscured) region of the star as:

\begin{equation}
   F = \oiint\limits_{\mathrm{S}(x,y)} I(x,y) \ dS
\end{equation}
where the surface $S$ parametrizes the unobscured portion of the stellar disk and $I$ is the specific intensity at a point $(x,y)$ on the projected surface of the star. 

\citet{starry2019} showed that if the specific intensity $I(x,y)$ is expanded in terms of the spherical harmonics, this double integral can be written in a simpler form that is both computationally and numerically easier to evaulate. We summarize the method here, and refer readers to the full text for the details. 

We start with a vector representing the stellar intensity in terms of spherical harmonics $\mathbf{y}$, and transform it into Green's basis as in Eq.~21 of \citet{starry2019}. We then have: 

\begin{equation}
   F = \oiint\limits_{\mathrm{S}(x,y)} \mathbf{\tilde{g}}^\mathsf{T}(x,y) \ \mathbf{A} \ \mathbf{y}\ dS 
\end{equation}

The vector $\mathbf{y}$ and change of basis matrix $A$ are not dependant on $x$ and $y$ and can consequently be pulled out of the integral, leaving only the Green's basis in the surface integral. Using the vector function $\mathbf{G_n}$, defined as the anti-exterior derivative of Green's basis, we can write:

\begin{equation}
   \oiint\limits_{\mathrm{S}(x,y)} \mathbf{\tilde{g}}^\mathsf{T}(x,y)\ dS \\
   = \oint\limits_{\mathrm{S}(x,y)} \mathbf{\tilde{g}}^\mathsf{T}(x,y) \cdot d\mathbf{r}
\end{equation}
where $\mathbf{r}$ is a vector function along a closed boundary that parametrizes the integral (see Fig.~\ref{fig:integralbounds}). 

\begin{figure}[ht!]
    \script{integralbounds.py}
    \begin{centering}
        \includegraphics[width=\linewidth]{figures/oblate_planet.pdf}
        \caption{Boundary of the line integral in Eq.~
        }
        \label{fig:integral_bounds}
    \end{centering}
\end{figure}

\subsection{Limb darkening}
\subsection{Oblate Case}
\subsection{Prolate Case}
\section{Prospects for Detection with JWST and other Instruments}
\label{sec:jwstdetect}
\subsection{Cramer-Rao Recoverability Estimates}
\label{sec:rec}
\subsection{Fisher Forecasting of Existing Planet Population}
\label{sec:fisherforecasting}
\subsection{Impact of Limb Darkening}
\subsection{Impact on Atmospheric Retreivals}
\section{Implementation in \lowercase{\texttt{jax0planet}}}
\label{sec:jax0planet}
\section{Application to [test system]}
\label{sec:lctest}
\section{Discussion and Future Work}
\label{disc}
\subsection{Hierarchical Inference}
\bibliography{bib}

\end{document}
