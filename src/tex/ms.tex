% Define document class
\documentclass[twocolumn]{aastex631}
\usepackage{showyourwork}

% Begin!
\begin{document}

% Title
\title{Semi-analytic transit models for oblate planets in Jax}

% Author list
\author{Shashank Dholakia} \author{Shishir Dholakia}

% Abstract with filler text
\begin{abstract}

\end{abstract}

% Main body with filler text
\section{Introduction}
\label{sec:intro}


\section{Ellipsoidal Model}
\label{sec:model}
The transit model we use is based on the formalism of the \texttt{starry} framework \citep{starry2019}, which provides closed-form expressions for occultations involving circular bodies in projection with general surface maps expressed in terms of spherical harmonics. \citet{dholakia2022} described an extension of the \texttt{starry} framework for an oblate occulted body, such as a rapidly rotating star. Here we present an extension for the case of any occultor which is an ellipse in projection. 

\subsection{Oblate Case}
\subsection{Prolate Case}
\section{Prospects for Detection with JWST and other Instruments}
\label{sec:jwstdetect}
\subsection{Cramer-Rao Recoverability Estimates}
\label{sec:rec}
\subsection{Fisher Forecasting of Existing Planet Population}
\label{sec:fisherforecasting}
\subsection{Impact of Limb Darkening}
\subsection{Impact on Atmospheric Retreivals}
\section{Implementation in \lowercase{\texttt{jax0planet}}}
\label{sec:jax0planet}
\section{Application to [test system]}
\label{sec:lctest}
\section{Discussion and Future Work}
\label{disc}
\subsection{Hierarchical Inference}
\bibliography{bib}

\end{document}
